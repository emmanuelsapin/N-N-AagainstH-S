\documentclass[preview]{standalone}
\usepackage{amsmath}
\usepackage{bm} % For bold math symbols (vectors and matrices)

\begin{document}

% --- SECTION: JOINT DENSITY MODEL ---
% This equation represents the mixture of two multivariate normal distributions.
% Component 1 (\mu_1, \Sigma_1) typically represents the H-S class.
% Component 2 (\mu_2, \Sigma_2) typically represents the N/N-A class.
\section*{1. Joint Density Model for 4-D Feature Vectors}
\begin{equation}
f(\mathbf{x} | \boldsymbol{\Theta}) = \pi \mathcal{N}(\mathbf{x} | \boldsymbol{\mu}_1, \boldsymbol{\Sigma}_1) + (1 - \pi) \mathcal{N}(\mathbf{x} | \boldsymbol{\mu}_2, \boldsymbol{\Sigma}_2)
\end{equation}
% \pi is the mixing proportion (weight) of the first component.

% --- SECTION: CLASSIFICATION METRIC ---
% This utilizes the Bayes' theorem logic to calculate the posterior probability.
% It identifies the probability that a specific pair belongs to the H-S group 
% given their observed haplotype sharing (vector x).
\section*{2. Posterior Probability for H-S Classification}
\begin{equation}
P(\text{H-S} | \mathbf{x}) = \frac{\pi \phi(\mathbf{x} | \bm{\mu}_{\text{H-S}}, \bm{\Sigma}_{\text{H-S}})}{\pi \phi(\mathbf{x} | \bm{\mu}_{\text{H-S}}, \bm{\Sigma}_{\text{H-S}}) + (1-\pi) \phi(\mathbf{x} | \bm{\mu}_{\text{N/N-A}}, \bm{\Sigma}_{\text{N/N-A}})}
\end{equation}
% \phi denotes the probability density function (PDF) of the multivariate normal.

\end{document}
